\documentclass{article}
\usepackage{eecstex}

\renewcommand{\thesubsection}{\thesection.\arabic{subsection}}
\renewcommand{\thesubsubsection}{\thesection.\alph{subsection}}

\DeclareMathOperator{\rank}{rank}

\title{DATA 100 HW 01}
\author{Bryan Ngo}
\date{2021-08-29}

\begin{document}

\maketitle

\section{Calculus}

\begin{equation}
    \sigma(x) = \frac{1}{1 + e^{-x}}
\end{equation}

\subsection{}

\begin{theorem}
    Given the function \(\sigma(x)\), \(\sigma(-x) = 1 - \sigma(x)\).
\end{theorem}

\begin{proof}
    \begin{align}
        \sigma(-x) &= \frac{1}{1 + e^x} \\
        &= \frac{1 + e^x - e^x}{1 + e^x} \\
        &= 1 - \frac{e^x}{1 + e^x} \frac{e^{-x}}{e^{-x}} \\
        &= 1 - \frac{1}{1 + e^{-x}} = 1 - \sigma(x)
    \end{align}
\end{proof}

\subsection{}

\begin{theorem}
    Given the function \(\sigma(x)\), \(\diff{x} \sigma(x) = \sigma(x) (1 - \sigma(x))\).
\end{theorem}

\begin{proof}
    \begin{align}
        \diff{x} \sigma(x) &= \diff{x} (1 + e^{-x})^{-1} = e^{-x} (1 + e^{-x})^{-2} \\
        &= \frac{1}{1 + e^{-x}} \frac{e^{-x}}{1 + e^{-x}} \frac{e^x}{e^x} \\
        &= \sigma(x) \frac{1}{1 + e^x} = \sigma(x) (1 - \sigma(x))
    \end{align}
\end{proof}

\section{Minimization}

Finding the zeroes of \(f(c)\),
\begin{align}
    f(c) &= \frac{1}{n} \sum_{i = 1}^n (x_i - c)^2 \\
    f'(c) &= -\frac{2}{n} \sum_{i = 1}^n (x_i - c) = 0 \\
    \sum_{i = 1}^n (x_i - c) = \sum_{i = 1}^n x_i - nc &= 0 \implies c = \frac{1}{n} \sum_{i = 1}^n x_i
\end{align}
Then, by the second derivative test,
\begin{align}
    f''(c) &= -\frac{2}{n} \sum_{i = 1}^n -1 = 2
\end{align}
implying a concave up function, meaning our zero is in fact a minimum.

\section{Probability \& Statistics}

\subsection{}

The percent of surveyed US adults who had a great deal of confidence in both scientists and religious leaders is impossible to find with information, since we only know what people trusted one at a time.
To know who trusts both, we would need data for every individual and who they trusted a great deal, which is not given in the bar graph.

\subsection{}

\begin{align}
    \Pr(C) &= 0.01 \implies \Pr(C^\complement) = 0.99 \\
    \Pr(T \mid C) &= 0.8 \\
    \Pr(T \mid C^\complement) &= 0.096 \\
    \Pr(C \mid T) &= \frac{\Pr(C) \Pr(T \mid C)}{\Pr(C) \Pr(T \mid C) + \Pr(C^\complement) \Pr(T \mid C^\complement)} \\
    &= \frac{0.01 \cdot 0.8}{0.01 \cdot 0.8 + 0.99 \cdot 0.096} \approx 0.078
\end{align}

\subsection{}

The standard deviation is \(\sigma = 6.1\) since the inflection point of the histogram is at \(\pm \sigma\), and the inflection point is around \(6.1\).

\section{Linear Algebra}

\subsection{}

\begin{equation}
    \bm{A} =
    \begin{bmatrix}
        1 & 1 \\
        0 & 1
    \end{bmatrix} \implies \rank(\bm{A}) = 2
\end{equation}
The matrix is full rank.

\subsection{}

\begin{equation}
    \bm{B} =
    \begin{bmatrix}
        3 & 0 \\
        -4 & 0
    \end{bmatrix} \implies \rank(\bm{B}) = 1
\end{equation}
The matrix is not full rank, since \(\bm{v}_1 = 0\bm{v}_2\).

\subsection{}

\begin{equation}
    \bm{C} =
    \begin{bmatrix}
        0 & 5 & 10 \\
        1 & 0 & 10
    \end{bmatrix} \implies \rank(\bm{C}) = 2
\end{equation}
The matrix is not full rank, since \(\bm{v}_3 = 2\bm{v}_1 + 10\bm{v}_2\).

\subsection{}

\begin{equation}
    \bm{D} =
    \begin{bmatrix}
        0 & -2 & 2 \\
        2 & -2 & 4 \\
        3 & 5 & -2
    \end{bmatrix} \implies \rank(\bm{D}) = 2
\end{equation}
The matrix is not full rank, since \(\bm{v}_3 = \bm{v}_1 - \bm{v}_2\).

\end{document}
